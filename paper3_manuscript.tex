%\documentclass[draftclsnofoot,onecolumn]{IEEEtran}
\documentclass[3p]{elsarticle}
\usepackage{tipa}
\usepackage{amsmath}%math
\usepackage{amsthm}%proof
\usepackage{xcolor}%color
\usepackage{bm}
\usepackage{graphicx}
\usepackage{booktabs}
\usepackage{textcomp}
\usepackage[colorlinks,linkcolor=black,anchorcolor=black,citecolor=black]{hyperref}
\usepackage{amsfonts,amssymb}
\usepackage{color,soul}
\theoremstyle{plain}
\newtheorem{myas}{Assumption}
\newtheorem{mydef}{Definition}
\newtheorem{mylem}{Lemma}
\newtheorem{mythm}{Theorem}

%\theoremstyle{remark}
\theoremstyle{remark}
\newtheorem{myrem}{Remark}

\begin{document}
\begin{frontmatter}
\title{Dual terminal sliding mode control}
\author{Zhiqiang Ma}
\author{Guanghui Sun\corref{cor1}}
\ead{guanghuisun@hit.edu.cn}
\cortext[cor1]{Corresponding author}
\address{Research Institute of Intelligent Control and Systems, Harbin Institute of Technology, Harbin 150001, China}

\begin{abstract}
\end{abstract}
\begin{keyword}
\end{keyword}
\end{frontmatter}
\section{Introduction}
\section{Dual sliding mode control}
Consider a four-order nominal nonlinear system:
\begin{align}
\begin{split}
\dot x_1(t) &= x_2(t)\\
\dot x_2(t) &= f_1(\bm x,t)+g_1(\bm x,t)+b_1(\bm x,t)u_1(t)\\
\dot x_3(t) &= x_4(t)\\
\dot x_5(t) &= f_2(\bm x,t)+g_2(\bm x,t)+b_2(\bm x,t)u_2(t),\label{eq:four order system}
\end{split}
\end{align}
where $bm x = [x_1,x_2,x_3,x_4]^T$ is the system vector state; $f(\bm x,t)$ and $b(\bm x,t)\neq 0$ are smooth nonlinear functions with respect to $\bm x$; $u_i(t)$, $i=1,2$ stands for the scalar input; $g_i(\bm x,t)$, $i=1,2$  is a defined as a bounded nonlinear function which satisfies $\Vert g_i(\bm x,t)\Vert\le l_i$, $l_i>0$, meaning the uncertainty and disturbance of the system. For the system mentioned above, we can establish a desired dual sliding manifold as:
\begin{align}
\bm s(t) = [s_1(t),s_2(t)]^T,
\end{align}
where
\begin{align}
\begin{split}
s_1(t) &= x_2(t)+\vert x_1(t)\vert^{\alpha} sgn(x_1(t))+x_1^3(t)-x_3(t)\\
s_2(t) &= x_4(t)+\vert x_3(t)\vert^{\alpha} sgn(x_3(t))+x_3^3(t)+x_1(t),\label{eq:dual sliding manifold}
\end{split}
\end{align}
where $sgn(x)$ is the sign function, and $0<\alpha<1$ is a positive constant. For the sake of simplicity, the symbol $t$ is usually omitting in this paper, which does not cause any confusion, and it will appear once the derivation concerns for the factor of the time strictly, i.e., the desired sliding manifold $\bm s(t)$ is often abbreviated to $\bm s$.\par
There exist two lemmas about the finite convergence theory, providing convergent forms which contribute to the verification of the finite convergence about the dual sliding mode control.
\begin{mylem}
As described in \cite{moulay2006finite}, it is assumed that the continuous function $V(t)$ is positive-define, satisfying the following condition:
\begin{align}
\dot V(t)\le -dV^\beta(t)\quad\forall t\ge t_0, V(t_0)\ge 0,
\end{align}
where $d$ is a positive scalar, and $0<\beta<1$. Then the function $V(t)$ can converge to the origin for any given $t_0$, and typically the finite convergent time $t_r$ can be given as follows:
\begin{align}
t_r \le t_0+\frac{V^{1-\beta}(t_0)}{d(1-\beta)}.
\end{align}\label{lemma:1}
\end{mylem}
\begin{mylem}
Considering a two-order system:
\begin{align}
\begin{split}
\dot x_1&=-\vert x_1\vert^\alpha sgn(x_1)-x_1^3+x_2\\
\dot x_2&=-\vert x_2\vert^\alpha sgn(x_3)-x_2^3-x_1.\label{eq:normal sliding manifold}
\end{split}
\end{align}\par
Selecting $V(t) = \frac{x_1^2+x_2^2}{2}$ obtains the finite convergent time as follows \cite{moulay2006finite}:
%since $\vert x\vert-x^\alpha+\vert x\vert \ge 0,0<\alpha<1$
\begin{align}
%t_s\le t_0+\frac{2(x_1^2+x_2^2)^{1-\alpha}}{1-\alpha}
t_s\le t_0+\frac{(2V(t_0))^{\frac{1-\alpha}{2}}}{1-\alpha},
\end{align}
where the definitions of  $t_s$, $t_0$, $\alpha$ are similar to Lemma \ref{lemma:1}. Introducing the amplitude scalar $d$ into the Eqs.~(\ref{eq:normal sliding manifold}) obtains  more normal forms:
\begin{align}
\dot x_1&=-d(\vert x_1\vert^\alpha sgn(x_1)-x_1^3+x_2)\\
\dot x_2&=-d(\vert x_2\vert^\alpha sgn(x_2)-x_2^3-x_1),\label{eq:more normal sliding manifold}
\end{align}
the convergent time of which is
\begin{align}
t_s\le t_0+\frac{(2V(t_0))^{\frac{1-\alpha}{2}}}{d(1-\alpha)}.
\end{align}\label{lemma:2}
\end{mylem}
For stabilization of the four-order system shown as~Eq. (\ref{eq:four order system}), dual sliding mode control inputs can be given according the following theorem.
\begin{mythm}\label{theorem:1}Dual Sliding Mode (DSM) Control Design.
The four-order system (\ref{eq:four order system}) will reach the desired manifold $\bm s = 0$ shown in~Eq. (\ref{eq:dual sliding manifold}) in finite time. The states of the system will converge to zero along the manifold $\bm s=0$ in finite time, if the inputs are designed as follows:
\begin{align}
u_1 &= -b_1^{-1}(\bm x)(f_1(\bm x)+\alpha\vert x_1\vert^{\alpha-1}x_2+3x_1^2x_2-x_4+k_1sgn(s_1))\\
u_2 &= -b_2^{-1}(\bm x)(f_2(\bm x)+\alpha\vert x_3\vert^{\alpha-1}x_4+3x_3^2x_4+x_2+k_2sgn(s_2)),
\end{align}
where $k_i = l_i+\eta_i$, $i=1,2$; $\eta_i>0$ is a positive scalar value.
\end{mythm}
\begin{proof}
Consider the Lyapunov function $V=\frac{1}{2}{\bm s}^T\bm s$, the derivative of which with respect to time $\dot V = {\bm s}^T \bm {\dot s}$ follows
\begin{align}
\begin{split}
\dot V &= {\bm s}^T\dot{\bm  s}\\
&=s_1\dot s_1+s_2\dot s_2\\
&=s_1(f_1(\bm x)+g_1(\bm x)+b_1(\bm x)u_1+\alpha\vert x_1\vert^{\alpha-1}x_2+3x_1^2x_2-x_4)\\
&\quad +s_2(f_2(\bm x)+g_2(\bm x)+b_2(\bm x)u_2+\alpha\vert x_3\vert^{\alpha-1}x_4+3x_3^2x_4+x_2).\label{eq:V_1}
\end{split}
\end{align}
Substituting inputs (\ref{eq:dual sliding manifold}) in~Eq. (\ref{eq:V_1}) yields
\begin{align*}
\dot V &= s_1(-k_1sgn(s_1)+g_1(\bm x))+s_2(-k_2sgn(s_2)+g_2(\bm x))\\
&= -\eta_1s_1sgn(s_1)-\eta_2s_2sgn(s_2)+(g_1(\bm x)s_1 - l_1s_1sgn(s_1))+(g_2(\bm x)s_2 - l_2s_2sgn(s_2))\\
&= -\eta_1\vert s_1\vert-\eta_2\vert s_2\vert+(g_1(\bm x)s_1 - l_1\vert s_1\vert)+(g_2(\bm x)s_2 - l_2\vert s_2\vert)\\
&\le - \min(\eta_1,\eta_2)(\vert s_1\vert+\vert s_2\vert)\\
&\le -\min(\eta_1,\eta_2)({\bm s}^T\bm s)^\frac{1}{2}\\
&\le -\sqrt{2}\min(\eta_1,\eta_2)V^\frac{1}{2}\le 0.
\end{align*}
Hence, the system (\ref{eq:four order system}) asymptotically converges to the desired manifold (\ref{eq:dual sliding manifold}), and incidentally the convergent time can be calculated according to the finite time convergence theory.\par
Once the system (\ref{eq:four order system}) reaches the desired manifold~(\ref{eq:dual sliding manifold}), the whole system is handled by the reduced system:
\begin{align*}
\dot x_1&=-\vert x_1\vert^\alpha sgn(x_1)-x_1^3+x_3\\
\dot x_3&=-\vert x_3\vert^\alpha sgn(x_3)-x_3^3-x_1.
\end{align*}\par
The total convergent time can be calculated by dividing the convergence into two phase with respect to the relation between the system and the manifold, namely, the approaching phase and the sliding phase, and specifically, according to Lemma \ref{lemma:2}, the total finite convergent time $t_f$ can be given out directly:
\begin{align}
t_f = \underbrace{t_0+\frac{\sqrt{2}V^{\frac{1}{2}}(t_0)}{\min(\eta_1,\eta_2)}}_{t_r}+\underbrace{\frac{(2\hat V(t_r))^{\frac{1-\alpha}{2}}}{1-\alpha}}_{t_s}<\infty,
\end{align}
where $\hat V(t) = \frac{x_1^2+x_3^2}{2}$ is the Lyapunov function describing the convergent motion of the reduced system. The approaching and sliding time are defined as $t_r$ and $t_s$, respectively. This completes the proof.
\end{proof}
On the basis of Theorem \ref{theorem:1}, we can develop the parameters in the manifold to improve the performance, which introduces a modified method for stabilizing the system~(\ref{eq:four order system}), and furthermore this method makes the design of the manifold more flexible. Considering the system~(\ref{eq:four order system}) with the manifold manifold $\bm s = [s_1,s_2]^T$, the details of which can be described by following expressions:
\begin{align}
\begin{split}
s_1(t) &= x_2(t)+\gamma(\vert x_1(t)\vert^{\alpha} sgn(x_1(t))+x_1^3(t))-\lambda x_3(t)\\
s_2(t) &= x_4(t)+\gamma(\vert x_3(t)\vert^{\alpha} sgn(x_3(t))+x_3^3(t))+\lambda x_1(t),\label{eq:modified dual sliding manifold}
\end{split}
\end{align}
where $\gamma$ is a positive constant and $\lambda\in R$ is a regulable parameter. Hence, the modified dual sliding mode control inputs are formed in the following theorem.
\begin{mythm}\label{theorem:2}Modified Dual Sliding Mode (MDSM) Control Design.
The four-order system (\ref{eq:four order system}) will reach the desired manifold $\bm s = 0$ shown in~Eq. (\ref{eq:modified dual sliding manifold}) in finite time. The states of the system will converge to zero along the manifold $\bm s=0$ in finite time, if the inputs are designed as follows:
\begin{align}
\begin{split}
u_1 &= -b_1^{-1}(\bm x)(f_1(\bm x)+\gamma\alpha\vert x_1\vert^{\alpha-1}x_2+3\gamma x_1^2x_2-\lambda x_4+k_1sgn(s_1))\\
u_2 &= -b_2^{-1}(\bm x)(f_2(\bm x)+\gamma\alpha\vert x_3\vert^{\alpha-1}x_4+3\gamma x_3^2x_4+\lambda x_2+k_2sgn(s_2)),\label{eq:modified input}
\end{split}
\end{align}
where $k_i = l_i+\eta_i$, $i=1,2$; $\eta_i>0$ is a positive scalar value. The total finite convergent time can be described as:
\begin{align}
t_f = \underbrace{t_0+\frac{\sqrt{2}V^{\frac{1}{2}}(t_0)}{\min(\eta_1,\eta_2)}}_{t_r}+\underbrace{\frac{(2\hat V(t_r))^{\frac{1-\alpha}{2}}}{\gamma(1-\alpha)}}_{t_s}<\infty,\label{eq:total convergent time}
\end{align}
\end{mythm}
\begin{proof}
The proof is similar to Theorem \ref{theorem:1}, hence it's omitted here.
\end{proof}
After introducing the parameters $\lambda$ and $\gamma$ into the DSM scheme, the performances of the approaching and sliding phases become regulatable, meaning that these parameters can adjust the convergent time analytically. Although $\lambda$ does not appear in Eq.~(\ref{eq:total convergent time}), the establishment of $V(t_0)$ involves it when the state $\bm x(t_0)$ has been fixed. Strictly speaking, it affects the approaching distance between $V(t_0)$ and $V(t_r)$, and normally $V(t_r)$ stands for the reaching status at $t_r$, namely, $V(t)=\bm s^T(t)\bm s(t) = 0,\forall t\ge t_r$. Admittedly, the variety of $\lambda$ is only modestly helpful in this approaching time regulation due to its limited effect on $V(t_0)$. $\eta$ is said to be an ''approaching speed'' according to its function, a suitable selection of which will adjust the approaching time as desired in some extent. Hence, $\lambda$ cooperates with $\eta$ to manage the approaching status, and by utilizing the similar analysis, owing to $\gamma$ adjusting the speed of sliding phase, it can be defined as the management parameter of the sliding status.
\begin{myrem}
For the proposed designs, the parameter $\gamma$ can be defined as different scalar for the individual input, if so, one can produce the total convergent time as:
\begin{align}
t_f = \underbrace{t_0+\frac{\sqrt{2}V^{\frac{1}{2}}(t_0)}{\min(\eta_1,\eta_2)}}_{t_r}+\underbrace{\frac{(2\hat V(t_r))^{\frac{1-\alpha}{2}}}{\min(\gamma_1,\gamma_2)(1-\alpha)}}_{t_s}<\infty,\label{eq:normal total convergent time}
\end{align}
with $\gamma_i$ corresponding to the original parameter $\gamma$ in input $u_i$.
\end{myrem}
The finite convergence of the DSM and MDSM scheme has been verified clearly, but similar to the conventional terminal sliding mode control, there exists another obstacle in the design mentioned above, namely, the issue of the singularity in control inputs. Essentially, the singularity occurs in Eq.~(\ref{eq:modified input}) when $x_1=0$ but $x_2\neq 0$ in $u_1$ (the phenomena happens to $u_2$ under the same condition). For coping with the singularity, a nonsingular MDSM scheme is presented after introducing an analytical technique into the MDSM scheme based on the following lemma \cite{feng2013nonsingular}.
\begin{mylem}\label{lemma:3}
Considering the four-order system (\ref{eq:four order system}), the terminal sliding mode manifold can be designed as:
\begin{align*}
s_1(t) &= x_2(t)+\gamma_1\vert x_1\vert^{\alpha_1}sgn(x_1)\\
s_2(t) &= x_4(t)+\gamma_2\vert x_3\vert^{\alpha_2}sgn(x_3),
\end{align*}
and nonsingular inputs are given out directly as follows:
\begin{align*}
u_1 &= b^{-1}_1(\bm x)(-f_1(\bm x)+sat(u_{f1},u_{s1})-k_1sgn(s_1))\\
u_2 &= b^{-1}_2(\bm x)(-f_2(\bm x)+sat(u_{f2},u_{s2})-k_2sgn(s_2)),
\end{align*}
in which
\begin{align*}
u_{f1}&=-\gamma_1\alpha_1\vert x_1\vert^{\alpha_1-1}x_2\\
u_{f2}&=-\gamma_2\alpha_2\vert x_3\vert^{\alpha_2-1}x_4,
\end{align*}
where $sat(u_f,u_s)$ is a saturation function:
\begin{align}
sat(u_f,u_s)=
\begin{cases}
u_s\quad &u_f\ge u_s\\
u_f\quad &-u_s\le u_f< u_s\\
-u_s\quad &u_f\le -u_s
\end{cases},
\end{align}
and $u_s>0$ can be determined according to the requirements of the performance.\par
Adopting the control scheme mentioned above, the system~(\ref{eq:four order system}) will converge to zero in finite time without any singularity occurring during the regulation process.
\end{mylem}
\begin{mythm}\label{theorem:3}
Consider a four-order system (\ref{eq:four order system}), and for eliminating the singularity in inputs of the MDSM scheme, select a new inputs to complete the nonsingular MDSM scheme. Introducing the saturation terms into the original inputs yields:
\begin{align}
u_1 &= -b_1^{-1}(\bm x)(f_1(\bm x)-sat(u_{f1},u_{s1})+3\gamma x_1^2x_2-\lambda x_4+k_1sgn(s_1))\\
u_2 &= -b_2^{-1}(\bm x)(f_2(\bm x)-sat(u_{f2},u_{s2})+3\gamma x_3^2x_4+\lambda x_2+k_2sgn(s_2)),\label{eq:nonsingular modified input}
\end{align}
one can obtain a nonsingular solution to the stabilization of the system~(\ref{eq:four order system}), providing the finite time convergence by using sliding mode technique.
\end{mythm}
\begin{proof}
The proof is similar to the graphical analysis mentioned in \cite{feng2013nonsingular}, and omitted here.
\end{proof}
\section{Dual sliding mode control for rigid manipulators}
In this section, a dual sliding mode control is proposed for the rigid $2$-link robot manipulator
\begin{align}
M(\bm q)\ddot {\bm q}+C(\bm q,\dot {\bm q})+g(\bm q)= \bm\tau(t)+{\bm d}(t),\label{eq:lagrangian system}
\end{align}
where $\bm q(t) = $ is the $2\times 1$ vector of joint angular position; ${M(\bm q)}$ is the $2\times 2$ inertia matrix; ${C(\bm q,\dot{\bm q})}$ denotes the Coriolis and centrifugal forces by using the $n\times1$ vector; ${g(\bm q)}$ stands for the $n\times 1$ gravitational torque; $\bm{\tau}(t)$ is referred to as the joint torques which are the inputs practically; $\bm d(t)$ expresses the $n\times 1$ bounded input disturbances vector. Normally, there exist some uncertainties in the manipulator, producing the dynamic equations:
\begin{align}
(M_0(\bm q)+\Delta {M(\bm q))\ddot {\bm q}}+{C_0(\bm q,\dot {\bm q})+\Delta C(\bm q,\dot {\bm q})}+g_0(\bm q)+\Delta g(\bm q)=\bm {\tau}(t)+\bm{d}(t),
\end{align}
where $M_0(\bm q)$, $C_0(\bm q,\dot {\bm q})$ and $g_0(\bm q)$ are the estimated terms affected by the the uncertainties including $M_0(\bm q)$, $C_0(\bm q,\dot {\bm q})$, $g_0(\bm q)$, and the disturbance $\bm{d}(t)$. Particularly, the following assumption is exhibited to explain the bounded characteristics of the robot dynamics \cite{feng2002non}:
\begin{align*}
&\Vert M(\bm q)\Vert< \theta_0\\
&\Vert C(\bm q,\dot {\bm q})\Vert< \phi_0+\phi_1\Vert\bm q\Vert+\phi_2\Vert\dot{\bm q}\Vert^2\\
&\Vert g(\bm q)\Vert< \psi_0+\psi_1\Vert\bm q\Vert\\
&\Vert \bm {\rho}(t)\Vert< \mu_0+\mu_1\Vert\bm q\Vert+\mu_2\Vert\dot{\bm q}\Vert^2,
\end{align*}
where $\bm {\rho}(t) = -\Delta M(\bm q)-\Delta C(\bm q,\dot {\bm q})-\Delta g(\bm q)$, and $\theta_0$, $\phi_0$, $\phi_1$, $\phi_2$, $\psi_0$, $\psi_1$, $\mu_0$, $\mu_1$, $\mu_2$ are positive constants.\par
Select $\bm q_d$ as the desired state of the Lagrangian system~(\ref{eq:lagrangian system}), and the derivative of $\bm q_d$ is designated as $\dot {\bm q}_d$. Hence the state error can be given out intuitively:
\begin{align}
\bm \varepsilon = \bm q -\bm q_d
\end{align}
With the purpose of simplifying the proceeding derivation, some symbol definitions for vector operating are presented in advance. For expressing the $n$-dimension vector including a sort of compound functions, the following symbol is introduced:
\begin{align*}
\lceil\bm h(t)\bm p(t)+\bm v(t)\rfloor = (h_1(t)p_1(t)+v_1(t),h_2(t)p_2(t)+v_2(t),\ldots,h_n(t)p_n(t)+v_n(t))^T.
\end{align*}
On the basis of the above results, one can design the dual sliding mode scheme for $2$-link robot manipulator by utilizing the following theorem.
\begin{mythm}\label{theorem:4}
For the rigid $2$-link manipulator~(\ref{eq:lagrangian system}), if the dual sliding mode manifold is selected as
\begin{align}
\bm s = \dot{\bm \varepsilon}+C_\gamma\lceil\bm{sgn}^\alpha(\bm \varepsilon)+\bm\varepsilon^3\rfloor+C_\lambda{\bm\varepsilon},\label{eq:lagrangian manifold}
\end{align}
where
\begin{align}
\begin{split}
&C_\gamma=diag[\gamma_1,\gamma_2]\\
&C_\lambda=
\begin{pmatrix}
0 &-\lambda\\ \lambda &0
\end{pmatrix}\\
&{sgn}^\alpha(\varepsilon)=\vert\varepsilon\vert^\alpha sgn(\varepsilon),
\end{split}
\end{align}
furthermore, one can obtain the corresponding input as follows to stabilize the system~(\ref{eq:lagrangian system}) as desired in finite time.
\begin{align}
\begin{split}
\bm\tau &= \bm u_0+\bm u_1 +\bm u_2\\
\bm u_0 &= -M_0(\bm q)C_r\lceil\alpha\vert\bm\varepsilon\vert^{\alpha-1}\dot{\bm \varepsilon}+3\bm \varepsilon^2\dot{\bm \varepsilon}\rfloor-M_0(\bm q)C_\lambda\dot{\bm \varepsilon}\\
\bm u_1 &= M_0(\bm q)\ddot {\bm q}_d+C_0(\bm q,\dot {\bm q})+g_0(\bm q)\\
\bm u_2 &= -\frac{(\bm s^TM_0^{-1}(\bm q))^T}{\Vert\bm s^TM_0^{-1}(\bm q)\Vert}(\mu_0+\mu_1\Vert\bm q\Vert+\mu_2\Vert\dot{\bm q}\Vert^2).
\end{split}
\end{align}
in which it is assumed that the parameters $\mu_0$, $\mu_1$ and $\mu_2$ are known.
\end{mythm}
\begin{proof}
Select the Lyapunov function
\begin{align}
V=\frac{1}{2}\bm s^T\bm s,
\end{align}
the derivative of which can be described as
\begin{align}
\begin{split}
\dot V &= \bm s^T\dot{\bm s}=\bm s^T(\ddot{\bm \varepsilon}+C_r\lceil\alpha\vert\bm\varepsilon\vert^{\alpha-1}\dot{\bm \varepsilon}+3\bm \varepsilon^2\dot{\bm \varepsilon}\rfloor+C_\lambda\dot{\bm \varepsilon})\\
&=\bm s^T(M_0^{-1}(\bm q)(-M_0(\bm q)\ddot {\bm q}_d-C_0(\bm q,\dot {\bm q})-g_0(\bm q)+\bm\tau+\bm\rho(t))\\
&\quad+C_r\lceil\alpha\vert\bm\varepsilon\vert^{\alpha-1}\dot{\bm \varepsilon}+3\bm \varepsilon^2\dot{\bm \varepsilon}\rfloor+C_\lambda\dot{\bm \varepsilon})\\
&=\bm s^T(M_0^{-1}(\bm q)(\bm u_0+\bm u_2+\bm\rho(t))+C_r\lceil\alpha\vert\bm\varepsilon\vert^{\alpha-1}\dot{\bm \varepsilon}+3\bm \varepsilon^2\dot{\bm \varepsilon}\rfloor+C_\lambda\dot{\bm \varepsilon})\\
&=\bm s^TM_0^{-1}(\bm q)(-\frac{(\bm s^TM_0^{-1}(\bm q))^T}{\Vert\bm s^TM_0^{-1}(\bm q)\Vert}(\mu_0+\mu_1\Vert\bm q\Vert+\mu_2\Vert\dot{\bm q}\Vert^2)+\bm\rho(t))\\
&\le -(\mu_0+\mu_1\Vert\bm q\Vert+\mu_2\Vert\dot{\bm q}\Vert^2)\Vert M_0^{-1}(\bm q)\Vert\Vert \bm s\Vert+\Vert\bm\rho(t)\Vert\Vert M_0^{-1}(\bm q)\Vert\Vert \bm s\Vert\\
&\le -(\mu_0+\mu_1\Vert\bm q\Vert+\mu_2\Vert\dot{\bm q}\Vert^2-\Vert\bm\rho(t)\Vert)\Vert M_0^{-1}(\bm q)\Vert\Vert \bm s\Vert.
\end{split}
\end{align}
The above result means that
\begin{align}
\dot V \le \varrho\Vert\bm s\Vert<0,
\end{align}
where
\begin{align}
\varrho=(\mu_0+\mu_1\Vert\bm q\Vert+\mu_2\Vert\dot{\bm q}\Vert^2-\Vert\bm\rho(t)\Vert)\Vert M_0^{-1}(\bm q)\Vert>0
\end{align}
and $\Vert\bm s\Vert\neq 0$, hence, the Lyapunov stability has been proved, meaning that the system will reach the desired manifold~(\ref{eq:lagrangian manifold}) in finite time. Once the reaching condition is satisfied, one can achieve a reduced system by substituting $\bm s = 0$ into the system~(\ref{eq:lagrangian system}). It owns the form, similar to Eq.~(\ref{eq:normal sliding manifold}), and therefore, one can adopt Theorem \ref{theorem:2} to drive the reduced system converging to the origin in finite time. The proof is completed.
\end{proof}
\begin{myrem}
At this point, the dual sliding mode control scheme for $2$-link robot manipulator has been demonstrated, and this technique is able to be extended to $n$-link manipulator. Consider a $n$-link robot manipulator in where $n$ is an even number, and the desired sliding manifold can be designed as
\begin{align}
\bm s = \dot{\bm \varepsilon}+C_{\gamma n}\lceil\bm{sgn}^\alpha(\bm \varepsilon)+\bm\varepsilon^3\rfloor+C_{\lambda n}{\bm\varepsilon},
\end{align}
where
\begin{align}
\begin{split}
&C_{\gamma n}=diag[\gamma_1,\gamma_2,\ldots,\gamma_n]\\
&C_{\lambda n}=
\begin{pmatrix}
\begin{pmatrix}
0 &-\lambda\\ \lambda &0
\end{pmatrix} &O_{2\times 2} & & &\\
O_{2\times 2} &\begin{pmatrix}
0 &-\lambda\\ \lambda &0
\end{pmatrix} & & &\\
& &\ddots & &\\
& & &\begin{pmatrix}
0 &-\lambda\\ \lambda &0
\end{pmatrix} &O_{2\times 2}\\
& & &O_{2\times 2} &\begin{pmatrix}
0 &-\lambda\\ \lambda &0
\end{pmatrix}
\end{pmatrix}.
\end{split}
\end{align}\par
Referring to Theorem \ref{theorem:4}, one can obtain the dual sliding mode control input for the even $n$-link manipulator. Essentially, this technique requires that the sliding mode manifold has to consist of $n/2$ pairs of individual sliding surfaces, in another words, this design for the manifold can be regarded as a work for yielding manifolds of the $n/2$ sub-systems.\par
Considering the situation that $n$ is an odd number, the $n$-link manipulator as we know is assumed as a full-actuated system, and hence an individual dimension can be excluded by using conventional terminal sliding mode control, which results in that the remained $n-1$-link manipulator can be controlled by the proposed DSM scheme.
\end{myrem}
\begin{myrem}
The singularity of the DSM scheme for the robot manipulator can be eliminated by using the technique introduced in Theorem \ref{theorem:3}.
\end{myrem}
\begin{myrem}
One can use the term $x^{a/b}$ instead of $sgn^\alpha(x)$ in the dual sliding mode control mentioned above, where $a$ and $b$ are odd numbers with $0<a/b<1$, as the entire proof maintains available theoretically.
\end{myrem}
\section{Simulation studies}
In this section, two studies will be presented, namely, a verification on the effectiveness of the proposed DSM and MDSM schemes and an application to a manipulator stabilization evaluating the performance of the control schemes.
\subsection{Effectiveness study}
For verifying the effectiveness of the proposed control scheme, consider the following four-order nonlinear system with the disturbance:
\begin{align}
\begin{split}
\dot x_1 &= x_2\\
\dot x_2 &= x_1^3+0.1\sin(20t)+u_1\\
\dot x_3 &= x_4\\
\dot x_4 &= 0.3\sin(3t)+0.5e^{-3t}+u_2,\label{eq:simulation four order system}
\end{split}
\end{align}
and one can produce a DSM control scheme conveniently. It is obvious that the disturbance in the system~(\ref{eq:simulation four order system}) behaves bounded, which is said to be the following uncertainty terms:
\begin{align*}
\Vert g_1(x,t)\Vert &= \Vert 0.1\sin(20t)\Vert\le 0.1\\
\Vert g_2(x,t)\Vert &= \Vert 0.3\sin(3t)+0.5e^{-3t}\Vert\le 0.8,
\end{align*}
and hence, based on the analysis in Theorem~\ref{theorem:1}, $\eta_1>0.1$ and $\eta_2>0.8$ can be determined accordingly to guarantee the reaching condition of the desired sliding manifold. Also, the inputs $u_1$ and $u_2$ are able to be viewed as virtual input terms which consist of the linear combination of the crucial inputs in system, i.e., for the system~(\ref{eq:simulation four order system}), the input terms can be expressed by $\bm u = \Xi\bm\tau$ particularly, where $\bm u\in R^2$ is the virtual input vector; $\bm \tau\in R^2$ stands for the crucial input vector; $\Xi\in R^{2\times 2}$ must have full rank.
\subsection{Performance of the robot manipulator}
In order to evaluate the performance of the proposed control scheme, a simulation about the $2$-link robot manipulator is presented in the following. The dynamics of this manipulator can be explained by the dynamic equations in following forms:
\begin{align}
\begin{bmatrix}
m_{12}(q_2) &m_{12}(q_2)\\
m_{21}(q_2) &m_{22}
\end{bmatrix}
\begin{pmatrix}
\ddot q_1\\
\ddot q_2
\end{pmatrix}
+
\begin{bmatrix}
-c(q_2)\dot q_1^2-2c(q_2)\dot q_1\dot q_2\\
c(q_2)\dot q_2^2
\end{bmatrix}+
\begin{bmatrix}
\xi_1(q_1,q_2) g\\
\xi_2(q_1,q_2) g
\end{bmatrix}=
\begin{pmatrix}
\tau_1\\
\tau_2
\end{pmatrix},
\end{align}
where
\begin{align*}
&m_{11}(q_2)=(m_1+m_2)r_1^2+m_2r_2^2+2m_2r_1r_2\cos(q_2)+J_1,\\
&m_{12}(q_2)=m_2r_2^2+m_2r_1r_2\cos(q_2),\\
&m_{21}(q_2)=m_{12}(q_2),\\
&m_{22}=m_2r_2^2+J_2,\\
&c(q_2)=m_2r_1r_2\sin(q_2),\\
&\xi_1(q_1,q_2) =((m_1+m_2)r_1\cos(q_2)+m_2r_2\cos(q_1+q_2)),\\
&\xi_2(q_1,q_2) = m_2r_2\cos(q_1+q_2).
\end{align*}\par
The values of the parameters are listed here, $r_1=1m$, $r_2=0.8m$, $J_1=5 kgm$, $J_2=5kgm$, $m_1=0.5kg$, $m_2=1.5kg$.
\section{Acknowledgment}
This work is partially supported by the National Natural Science Foundation of China (No. 61104112, 61503097).
\section{References}
\bibliography{paper3_ref}
\bibliographystyle{elsarticle-num}
\end{document}
